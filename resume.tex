\documentclass[11pt]{article}

\usepackage{times,mathptm}
\usepackage{pifont}
\usepackage{exscale}
\usepackage{latexsym}
\usepackage{amsmath}
\usepackage{epsfig}

\textwidth 6.5in
\textheight 9in
\oddsidemargin -0.1in
\topmargin -0.6in

\parindent 0pt
\parskip 0pt

\newcommand{\headerrow}[2]
{\begin{tabular*}{\linewidth}{l@{\extracolsep{\fill}}r}
	#1 &
	#2 \\
\end{tabular*}}

\begin{document}

\begin{center}
	\begin{huge}
		\bf Jack Weiss
	\end{huge}

	66 Woodward Street, San Francisco, CA 94103 \\
	415-715-7575 \textbullet\ jacobaweiss@gmail.com
\end{center}
\hrule
\vspace{0.5em}

%Education section
\vspace{0.8em}
\begin{LARGE}
	\bf Education
\end{LARGE}
\vspace{0.5em} \\
\headerrow
	{\begin{Large}The College of William \& Mary\end{Large}}
	{Williamsburg, VA}
\headerrow
	{\textit{B.S. Computer Science}}
	{\textit{2009-2013}}
%coursework
\vspace{-1.5em}
\begin{center}
	\begin{tabular}{lcr}
		& {\bf Coursework}  & \\
		\textit{Algorithms} & \textit{Software Development} & \textit{Data Structures} \\
		\textit{Database Design} & \textit{Programming Languages} & \textit{Web Design}

	\end{tabular}
\end{center}
\hrule
\vspace{0.5em}

%Work Experience
\vspace{0.8em}
\begin{LARGE}
	\bf Experience
\end{LARGE}
\vspace{0.5em}

	%indiegogo
	\headerrow
		{\begin{Large}Indiegogo\end{Large}}
		{}
	\headerrow
		{\textit{Software Engineer}}
		{\textit{August 2014 - February 2016}}
	\begin{itemize}
		\item Lead development and project management on a new in-house A/B testing framework. The system is currently managing numerous active split tests, and corresponding assignments for millions of site visitors.
		\vspace{-0.8em}
		\item Lead redesign efforts of the email system, including the creation of an email preference center used to improve user retention and reengineering the delivery architecture to be more usable by the engineering and marketing teams.
		\vspace{-0.8em}
		\item Engineered "Chat Ops" integration for the developer chat room, including automated notifications for continuous integration, Github notifications and pull request integration, and staging server management.
	\end{itemize}

	%Lumos Labs
	\headerrow
		{\begin{Large}Lumos Labs\end{Large}}
		{}
	\headerrow
		{\textit{Junior Software Engineer}}
		{\textit{June 2013 - August 2014}}
	\begin{itemize}
		\item Designed and implemented the Family Plan invitation manager and weekly family challenge features, leveraging Backbone to create a fast and enjoyable experience for subscribers.
		\vspace{-0.8em}
		\item Designed, developed, and open-sourced the inline validation system used on the sign-up page and across the website ({\textit{see github.com/lumoslabs/comply}})
		\vspace{-0.8em}
		\item Managed the payment and user-authentication systems. This included additional support for new international currencies and laws, and redesigning payment system processing to support the Google Play store.
	\end{itemize}

	%viget
	\headerrow
		{\begin{Large}Viget Labs\end{Large}}
		{}
	\headerrow
		{\textit{Ruby on Rails Intern}}
		{\textit{Summer, 2012}}
	\begin{itemize}
		\item Developed Ruby applications with Ruby-on-Rails and Sinatra, using front-end tools such as Twitter Bootstrap, and testing tools such as Rspec and Simplecov.
		\vspace{-0.8em}
		\item Learned principles and techniques such as pair-programming, test-driven development, and version control for managing projects.
		\vspace{-0.8em}
		\item Worked with back-end development team on existing projects, tackling problems including data migration, web application routing, and page caching.
	\end{itemize}

\hrule
\vspace{0.5em}

%Projects & Interests section
\vspace{0.8em}
\begin{LARGE}
	\bf Projects \& Other Interests
\end{LARGE}
\vspace{0.5em} \\
\headerrow
	{\begin{Large}Open-Source Contributions\end{Large}}
	{\textit{github.com/jacobaweiss}}
	\begin{normalsize}
	\hspace*{1.5em}I have created and contributed to numerous open-source projects. This ranges from Rails libraries like {\textit{Comply}}, which runs server-side form validations on the client, to chat room scripts like {\textit{Hubot-CircleCI-Notify}}, which automatically notifies you when the CI builds you watch are finished. Through these projects, I've experienced the learning and fun that goes along with programming, and have given back to the development community on which we all depend. \\
	\end{normalsize}

\headerrow
	{\begin{Large}Humorous Projects\end{Large}}
	{\textit{cageme.herokuapp.com}}
	\begin{normalsize}
	\hspace*{1.5em}In addition to {\textit{utterly serious}} open-source development, I have always made a point to learn new skills through humor. To explore the world of web development in Ruby, I built {\textit{CageMe}}, an image placeholder generator which exclusively uses images of Nicolas Cage's face doctored onto popular pictures. While learning about Bash scripting in school, I wrote {\textit{Celeryman}}, which, when ran, plays a continuous loop of a dancing Paul Rudd in your terminal. These have been great opportunities to explore new tools while making myself and my friends laugh. \\
	\end{normalsize}

\end{document}
\end{document}
